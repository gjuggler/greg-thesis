\begin{center}
\Large \mytitle

\large Summary
\end{center}

\hfill Gregory Jordan \\
\noindent{\today} \hfill Darwin College \\

Insight into the evolution of protein-coding genes can be gained from
the use of phylogenetic codon models. Recently sequenced mammalian
genomes and powerful analysis methods developed over the past decade
provide the potential to globally measure the impact of natural
selection on protein sequences at a fine scale. The detection of
positive selection in particular is of great interest, with relevance
to the study of host-parasite conflicts, immune system evolution, and
adaptive differences between species.

Our ability to make confident estimates of the prevalence of positive
selection in proteins has been hampered to some extent by uncertainty
regarding the level of false positives resulting from alignment
error. To this end, I conduct a simulation study to estimate the rate
of false positive results attributable to alignment error. A variety
of aligners and alignment filtering methods are compared, showing a
striking difference between different aligners in their tendency to
produce false positive results. Under most conditions, the best
aligners tend to produce very few false positives due to misalignment.

The rest of this thesis focuses on two genome-wide studies: an
analysis of sitewise selective pressures across 38 mammalian genomes,
and a genome-wide scan for genes with evidence of accelerated
evolution in gorilla and the African great apes. In the mammalian
analysis, the global distribution of sitewise evolutionary constraint
is characterized and strong evidence is presented for less positive
selection in the protein-coding genes of rodents compared to primates
and other mammalian orders. New methods are developed for combining
sitewise estimates across genes and protein-coding domains, revealing
widespread signals of positive selection in genes and domains related
to host defense and, surprisingly, centromere binding. The African
great ape analysis uses phylogenetic codon models to identify genes
which have experienced elevated evolutionary rates in gorilla, human
and chimpanzee. Similar numbers of accelerated genes are identified in
each of these genomes, and several accelerated genes are identified in
gorilla with plausible relationships to its unique phenotypic and
behavioral characteristics. Finally, genome-wide coding alignments are
used to infer genome-wide selective pressures along each branch of the
great ape tree, providing corroborating evidence of a trend towards
decreasing population sizes in the recent evolutionary history of
African great apes.
