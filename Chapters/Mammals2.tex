%************************************************
\chapter{The use of sitewise selective pressures to characterise the evolution of genes and domains in mammals}
%************************************************
\section{Introduction}

% From mammals paper supp. methods...
%https://docs.google.com/leaf?id=0B5HhSYGuKPKpYTY1YzVjNTgtODc2Yy00NzQ0LTgwZTQtMTQ3ZTA5ZWJjZGJm&hl=en_US

A clear definition of what constitutes a positively selected gene
(PSG) is elusive and depends heavily on the model the data. Neutral
theory, however, does provide a precise definition: the dN/dS ratio
gives a natural threshold for distinguishing purifying from positive
selection at dN/dS = 1, above which point positive selection can be
inferred. This leads logically to the most stringent definition of a
PSG as a gene whose overall mean dN/dS ratio is greater than
1. However, as noted by Yang [5], this definition of a PSG is
extremely conservative and is likely to be met by very few genes. To
examine this hypothesis in the context of mammalian proteins, we
averaged the site-wise dN/dS estimates from each gene tree tested to
obtain an average dN/dS value, mean-dN/dS. At a threshold of
mean-dN/dS = 1, 126 out of 15,451 gene trees analyzed showed evidence
for positive selection (0.82\%).  At a threshold of mean-dN/dS = 0.5,
1,438 genes were PSGs (9.3\%).

More powerful scans for positive selection relax the requirement that
the overall dN/dS of a gene is above 1 under the assumption that the
force of positive selection is unlikely to act along an entire protein
sequence and throughout the entire evolutionary history of a gene
tree. This led to models that allow for variation in the dN/dS ratio
along the sequence or along the phylogeny. If the data shows
statistically significant evidence for positive selection under a
given model, then it is considered a PSG – thus, the definition of a
PSG is tightly linked to the evolutionary model being applied and the
evolutionary clade or lineage being analyzed. In the case of the SLR
method (which allows for variation of dN/dS along the sequence but not
along the phylogeny), the most appropriate definition of a PSG is a
gene where one or more sites show statistical evidence for positive
selection after correcting for multiple tests.  With this definition,
we find 2,865 positively selected gene trees in mammals (18.5\%). We
may also formulate a more conservative definition in an attempt to
minimize the number of false positives resulting from a background of
reduced purifying constraint, considering genes with mean-dN/dS of
less than 0.5 and at least 2 positively-selected codons to be
PSGs. This definition yields 739 positively selected gene trees
(4.8\%).

Although the mean-dN/dS > 1 definition of a PSG is unreasonably
conservative (and, upon manual inspection, picks out many proteins
with especially severe alignment or gene annotation errors), the
numbers of PSGs resulting from both codon-based definitions applied to
the current data (4.8\% at the low end and 18.5\% at the high end) are
similar to the range of other recently published comparative scans for
positive selection, which identified between 3.3\% [6] and 16.6\% [7]
of genes as PSGs in mammals or primates.


\section{Comparison of sitewise results to previously described sets of positively selected genes}

\section{Using sitewise selective pressures to characterise the evolution of genes}
\subsection{Identifying genes subject to positive selection}
\subsection{Identifying genes subject to strong or weak purifying selection}

\section{Using sitewise selective pressures to characterise the evolution of protein domains}

\tocite{Moses2009}{They use residue probabilities to generate
  ``preferred residue'' predictions, and look at human polymorphism to
  assign whether a SNP is moving to or from a preferred residue.}

In order to evaluate the prevalence of positive selection in mammals
for various domain structures, we used the Pfam domain mappings from
the Ensembl database (release version 54) to annotate the site-wise
dN/dS values with domain assignments. We mapped Pfam protein domain
annotations from all sequences in a gene tree onto the alignment,
keeping only features with a hit score greater than 20 and alignment
sites with greater than 4 columns and an inferred dN/dS value of less
than 50. We then removed any domains with fewer than one thousand
annotated sites, to avoid errors resulting from small sample sizes.

The domain annotations were collated in a variety of ways: (1) taking
the mean omega value across all annotated sites [omega column in the
  table below], (2) counting the number of PSCs within all annotated
sites [psc], and (3) counting the number of PSCs per annotated site
[psc\_corr]. The len column represents the number of alignment sites
containing the given Pfam annotation. When the list of domains is
sorted by the total number of annotated PSCs we see the more familiar
positively-selected domains at the top of the list (immunoglobulin,
collagen), followed by more novel positively-selected domains such as
the 7 transmembrane receptor, protein kinase domain, and ion transport
protein.

\subsection{Identifying protein domains subject to positive selection}
\subsection{Identifying protein domains subject to strong or weak purifying selection}

\section{Identifying genes under unusual selective pressures in mammalian superorders}

% Don't forget to take a dig at Goodman et al. 2009 PNAS!  They
% interpret their results (higher dN/dS in primates / laur and higher
% mutation rate in rodents) as somehow indicating correlated evolution
% between primates and laurasiatheria
