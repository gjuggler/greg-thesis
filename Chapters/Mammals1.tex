%************************************************
\chapter{Patterns of sitewise selection in mammalian protein-coding genes}\label{ch:mammals1}
%************************************************

\section{Introduction}

\subsection{The Mammalian Genome Project}

\newcommand{\mgp}{Mammalian Genome Project\xspace}

A major goal of mammalian comparative genomics has been to quantify,
identify and understand the fraction of the human genome that is under
evolutionary constraint. The first non-human mammalian genomes showed
at least 5\% of the human genome to be under purifying selection
\citep{Mouse_Genome_Sequencing_Consortium2002,Gibbs2004,LindbladToh2005},
but the small number of genomes available limited the extent to which
regions of evolutionary constraint could be identified. The \mgp, a
coordinated set of genome sequencing projects organised by the Broad
Institute of MIT and Harvard, was designed with the primary purpose of
increasing the accuracy and confidence with which regions of the human
genome that have evolved under evolutionary constraint in mammals
could be identified \citep{TODO}.

The mammalian tree of like has a star-like shape, owing to the rapid
and extensive radiation of mammalian species that occurred starting \todo{XYZ mya}.

\subsection{The Sitewise Likelihood Ratio test}

\subsection{Data quality concerns: alignment and sequencing error}

\subsection{Gene trees, genomic alignments, and low-coverage genomes in the Ensembl database}

\section{Methods to identify orthologous subtrees within large mammalian gene families}

\section{Analysis of the genome-wide set of orthologous mammalian trees}

\section{Analysis of the global distribution of mammalian selective pressures}

\section{Analysis of sitewise estimates from three mammalian sub-clades}

\section{Evaluation of the effect of GC content, recombination rate, and codon usage on sitewise dnds estimates and the detection of positive selection}

\subsection{Mammalian sitewise selective pressures are not subject to strong effects of biased gene conversion}

\subsection{Mammalian sitewise selective pressures suggest increased efficacy of natural selection in regions of high recombination}

\section{Conclusions and future work}
