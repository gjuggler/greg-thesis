%************************************************
\chapter{Introduction}\label{ch_intro}
%************************************************

\section{Introduction}
\draft{Write an introductory page or two. Keep it broad, cite some old
  stuff.}

\section{Biological background}
  \label{bio_intro}
  \draft{All of evolutionary biology is about understanding the history
    of evolution, and is thus tied to the circumstances under which
    such evolution occurred. Thus, a brief overview of that history is
    relevant and useful in this thesis.}

  \subsection{The evolutionary history of vertebrates and mammals}
  \label{evolution_intro}
  \draft{Write a quick summary of the genome evolution of vertebrates
    and mammals. Mention 2R duplication, genome size growth,
    transposable elements.}

  \subsection{Mammalian population structure, adaptation and natural selection}
  \label{popgen_intro}
  \draft{Introduce the concept of adaptation (molecular
    vs. morphological / ecological), the varied behavioral
    characteristics of modern day mammals (focusing on mammalian
    superorders and great apes, as expanded in mammals and gorilla
    chapters), and the impact of population structure / population
    size on the efficacy of natural selection.}

\section{Mathematical methods for genomic analysis}
  \label{math_intro}
  \draft{Introduce the importance of imperfect replication / copying as
    the substrate of evolution and a very convenient phenomenon for
    mathematical analysis. The balance between randomness and
    structure in evolutionary models.}

  \subsection{Codon models of evolution}
  \label{codon_intro}
  \draft{Introduce the idea of modeling protein evolution as a markov
    process acting on codon sequences: the incorporation of
    mechanistic parameters for Ts:Tv bias (kappa), dN/dS ratio
    (omega), or empirical models a la . Talk about heterogeneity
    the idea that real data may strongly violate certain models.}

  \subsection{Hypothesis testing with likelihood ratio tests}
  \label{ml_intro}
  \draft{Briefly introduce the idea of nested models and likelihood
    ratio tests (used for PAML in Slrsim and Gorilla chapters, and for
    SLR in Mammals chapters)}

  \subsection{Detecting purifying and positive selection in protein-coding sequence}
  \label{pos-sel_intro}
  \draft{Briefly run through the history of detecting purifying /
    positive selection in genes and sites. Mention history of PAML
    models, alternative approaches, and fully describe SLR's
    approach.}

  \subsubsection{The Sitewise Likelihood Ratio test}
  \label{slr_intro}
  % From Slrsim Methods
  \draft{SLR implements a method specifically designed for sitewise estimates
  which has been shown in sim- ulations to perform as well as or
  better than PAML’s sitewise random sites models (Massingham and
  Goldman, 2005). SLR models codon evolution as a continuous-time
  Markov process where substitutions at one site are independent of
  substitutions at all other sites. No assumptions are made regarding
  the distribution of ω ratios within the alignment. The value of ω is
  considered to be an independent parame- ter at each site: after
  first optimizing shared parameters using the whole alignment, SLR
  uses the shared parameters and the data at each alignment site to
  calculate a sitewise statistic for non-neutral evolution.  This
  statistic is based on a likelihood-ratio test where the null model
  is neutral evolution (ω = 1) and the alternative model is either
  purifying or positive selection (ω < 1 or ω > 1, respectively). The
  raw statistic measures the strength of evidence for non-neutral
  evolution at each site; following Massingham and Goldman (2005) we
  use a signed version of the SLR statistic (created by negating the
  statistic for sites with ω < 1) as the test statistic for positive
  selection.}

  \subsection{Identifying biological trends in sets of genes and proteins}
  \label{go_intro}
  \draft{Introduce the Gene Ontology and Pfam databases, which annotate
    genes or components of genes with structured ontologies of
    functions or domains, respectively. Introduce the methods for
    detecting enriched GO terms. Note problems and biases involved in
    the basic methodology and describe algorithms / corrections
    introduced to correct for certain biases: topGO for hierarchical
    GO structure and goseq for element length.}

  \subsection{Correcting for multiple testing in genome-scale datasets}
  \label{bh_intro}
  \draft{Note the issue of correcting for multiple testing in
    genome-scale datasets. Clarify the differences between nominal
    p-value, family-wise error rate, FDR. Provide examples of when /
    where certain methods may be more applicable than others.}
